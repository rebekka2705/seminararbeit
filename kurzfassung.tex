In recent years web applications have grown not only in lines of code but also in complexity. With this complexity also the target surface for malicious attacks have grown. The different layers of an web application offer different entry points for attackers. As many application handle sensitive and confidential data it is vital that the developers secure their code against any possible exploitation. Implementing security features into the code require broad knowledge and time. Due to the enormous size of the source code it is impossible to check the code manually. Therefore automated security tools have been developed to support programmers. These security tools can be divided into three categories. First the static analysis tools that focus on the source code and scan for vulnerabilities without actually executing the application. The dynamic analysis tools in some cases are not provided with the source code instead they observe the application in execution. As both of those tools can not cover all possible attacks scenario more and more hybrid tools are introduced on the market to expand the scope of the analysis. 