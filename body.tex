%for reference to this section
\section{Introduction}
\label{section:Introduction}
In recent years, web applications have become more and more complex. Due to the growing size of applications it is difficult to satisfy all security requirements. Many applications handle confidential and sensitive data and have become a popular target for malicious attacks. A security breach could have severe consequences depending on the data that has been compromised.\newline The OWASP Foundation\footnote{ \url{https://owasp.org/www-project-top-ten/}} provides a top ten list of web application security risks. SQL Injection, cross-site scripting or sensitive data exposure are only a few of the many issues to consider while developing. But as the implementation of security measures are a very time-consuming job, many developers are lacking the time and/or knowledge to implement those security techniques. Many web applications deployed on the Internet contain severe security vulnerabilities. Almost half of the web applications reviewed by the Web Application Security Consortium contained vulnerabilities that are considered high risk \autocite[2]{Li2014}.
So it comes as no surprise that many security analysis tools have been developed to support developers to scan their applications, reveal bugs or to confirm the security measures they have implemented.\newline There are different types of web security tools. Static analysis tools assess the application without executing the program. A static analysis tool analyses the code to search for e.g. syntatic errors. The analysis considers all code execution paths. Dynamic analysis, on the other hand, evaluates the app's behaviour during execution. For this kind of analysis specific attack vectors can be used to validate the source code. Some tools combine those techniques \autocite[]{Lam2008,Hosek2011}.\newline In this paper, I will give an overview of different types of web security tools, starting with a more detailed description of the top-ten web security risks and the distinction between static and dynamic analysis in the first chapter. Afterwards, I will explain in detail how static analysis tools like RubyX, Derailer or SPACE analyse applications \autocite[]{Chaudhuri2010, Near2014, Near2016}. Another important topic in regards to analysis tools is symbolic execution. Many tools use this analysing technique that translates concrete values in the source code to symbolic values that are executed in an alternate runtime \autocite[]{Bocic2017,Near2016,Near2012}. Other tools combine the static and dynamic approach, like PQL or SafeWeb, or provide black box testing with no access to the source code \autocite[]{Lam2008,Hosek2011,Araujo2018}.




\section{Web Application Security Risks}
\autocite[]{ElIdrissi2017}, \autocite[]{Li2014}
\section{Static Analysis vs Dynamic Analysis}
\subsection{Static Analysis Tools}
\autocite[]{Chaudhuri2010}, \autocite[]{Near2014}, \autocite[]{Bocic2014}, \autocite[]{Munetoh2013a}, \autocite[]{Jovanovic2006}, \autocite[]{Munetoh2013}
\subsection{Symbolic Execution Tools}
\autocite[]{Chaudhuri2010}, \autocite[]{Near2014}, \autocite[]{Near2012}, \autocite[]{Nijjar2011}, \autocite[]{Near2016}, \autocite[]{Jackson2002}, \autocite[]{Bocic2016}, \autocite[]{Cadar2011}, \autocite[]{Bocic2014}
\subsection{Dynamic Analysis Tools}
\autocite[]{Yip2009}, \autocite[]{Felt2011}

\subsection{Hybrid Analysis Tools}
Static analysis and dynamic analysis have their own individual drawbacks. As static analysis scans the code without executing the program, this means it can analyse the application quicker. But this automatically limits the extent to which the analyser can check the application. Dynamically created content and functions can not be analysed and therefore some vulnerabilities remain undetected. On the other hand dynamic analysis tools often are not provided with the source code and are only monitoring the execution. They rely on attacks vectors to identify vulnerabilities. If those vectors are wrongly constructed this could lead to false positives.
Due to the complex nature of many modern web application hybrid analysis tools that combine static and dynamic elements try to limit the amount of undetected vulnerabilities \autocite[]{Araujo2018, Jahanshahi2018}. In this chapter I would like to provide an insight into recently developed hybrid analysis tools.

\subsubsection{SQLBlock}
\textcite[]{Jahanshahi2018} introduce in their recently published paper a tool called SQLBlock that secures a web application against various kinds of SQL Injections. Manipulating SQL queries can lead to serious database damage/manipulation or exposure of classified information to the attacker. An overview of the different categories of SQL attacks by \textcite[3ff.]{Halfond2008} states eight types.

\textbf{Tautology} \newline
\textbf{Union Query}\newline
\emph{Piggy-backed Query}
\emph{Alternate Encoding}


\autocite[]{Lam2008}, \autocite[]{Hosek2011} 


% h = try to place the figure Here
% t = try to place the figure at the Top of a page
% p = try to place this figure along with others on a separate Page
% Note that LaTeX has a sophisticated ranking algorithm to place figures.
% It is not always easy to accept LaTeX's placing but it is harder doing it
% manually. Just let it go ;-)

