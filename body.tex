%for reference to this section
\section{Introduction}
\label{section:Introduction}
In recent years, web applications have become more and more complex. Due to the growing size of applications it is difficult to satisfy all security requirements. Many applications handle confidential and sensitive data and have become a popular target for malicious attacks. A security breach could have severe consequences depending on the data that has been compromised.\newline
The OWASP Foundation\footnote{ \url{https://owasp.org/www-project-top-ten/}} provides a top ten list of web application security risks. SQL Injection, cross-site scripting or sensitive data exposure are only a few of the many issues to consider while developing. But as the implementation of security measures are a very time-consuming job, many developers are lacking the time and/or knowledge to implement those security techniques. Many web applications deployed on the Internet contain severe security vulnerabilities. Almost half of the web applications reviewed by the Web Application Security Consortium contained vulnerabilities that are considered high risk \autocite[2]{Li2014}.
So it comes as no surprise that many security analysis tools have been developed to support developers to scan their applications, reveal bugs or to confirm the security measures they have implemented.\newline There are different types of web security tools. Static analysis tools assess the application without executing the program. A static analysis tool analyses the code to search for e.g. syntatic errors. The analysis considers all code execution paths. Dynamic analysis, on the other hand, evaluates the app's behaviour during execution. For this kind of analysis specific attack vectors can be used to validate the source code. Some tools combine those techniques \autocite[]{Lam2008,Hosek2011}.\newline In this paper, I will give an overview of different types of web security tools, starting with a more detailed description of the top-ten web security risks and the distinction between static and dynamic analysis in the first chapter. Afterwards, I will explain in detail how static analysis tools like RubyX, Derailer or SPACE analyse applications \autocite[]{Chaudhuri2010, Near2014, Near2016}.  Other tools combine the static and dynamic approach, like PQL or SafeWeb, or provide black box testing with no access to the source code \autocite[]{Lam2008,Hosek2011,Araujo2018}.

\section{Web Application Security Risks}
As web applications are growing and becoming more complex also the target area for security attacks increases. Attackers adapt to the security measures developers have implemented and come up with new ways to infiltrate web applications.\newline
\textcite[5]{Li2014} divide the most common web security risks in three categories: input validation vulnerability, session management vulnerability and application logic vulnerability. This is a basic classification of security threats that offer a good overview of what aspects to consider while developing. The Open Web Application Security Project (OWASP) is a well-known resource for the latest and most common security risk for web apps.\footnote{ \url{https://owasp.org/www-project-top-ten/}} This list contains a more detailed description of specific threats. The top three leading issues of latest version of 2017 are: 

\begin{enumerate}
    \item \textbf{Injection}: untrusted data is injected into commands or queries to perform unintended actions
    \item \textbf{Broken Authentication}: Session management or authentication is compromised so that attackers can access user data or exploit the application with valid user credentials    
    \item \textbf{Sensitive Data Exposure}: lacking proper protection and encryption of sensitive data such as credit card or healthcare information can lead to exposure and misuse of user data
\end{enumerate}

Without doubt responsible software development has to keep these aspects in mind when developing modern web solutions. Unfortunately many further security topics have to be considered while developing. Broken access control, cross-site scripting or vulnerable libraries or frameworks offer target points for attacks. Even implemented security measures may pose a threat. Security misconfigurations can be entry points for attacks.\newline

Faced with these various threats a lot of research has gone into developing automated tools to protect modern web applications. Static analysis focuses on the source code and scans the application without executing the program. Dynamic analysis tools, on the other hand, observe applications in runtime to detect vulnerabilities and prevent attacks. As both tools on their own can't cover all security issues also more and more hybrid tools that combine both methods have been introduced in recent years \autocite[]{Jahanshahi2018, Lam2008, Hosek2011}.\newline 
In the following chapters I will introduce tools for each analysis approach. Each tool has its own target area and I would like to convey basic knowledge about the work flow of every approach. Due to the vast amount of tools on the market only a few tools can be presented in the course of this paper.


\section{Static Analysis Tools}
Web Applications have been becoming more complex and growing in lines of code. To ensure the quality of the code especially with a lot of engineers working on the code base it is necessary to review and analyse code not only in regards to quality aspects but also to detect potential security vulnerabilities. However the sheer amount of code makes it impossible to completely to rely on individual coders to review the code manually. Automated analysis tools provide the service to check code for various vulnerabilities and help to detect weaknesses in the code. So called static analysis tools run through the source code without executing the program. One downside of static analysis tools is that each programming language needs a specifically designed tool. So there are already many static analysis tools in use like Rubocop\footnote{ \url{https://github.com/rubocop-hq/rubocop}} for Ruby or Pixy, a tool for PHP \autocite[]{Jovanovic2006}. Other tools use symbolic execution %just mention the use of it with papers \newline

%\begin{comment}
%\autocite[]{Near2014} Derailer
%\autocite[]{Chaudhuri2010}, \autocite[]{Bocic2014}, \autocite[]{Munetoh2013a}, %\autocite[]{Jovanovic2006}, \autocite[]{Munetoh2013}
%\end{comment}


\textcite[]{Maskur2019} recently introduced a tool that uses static analysis combined with a taint analysis method for PHP applications. What is taint analysis? As \textcite[]{Shannon2018} explains a variable that is marked as tainted means it is potentially dangerous. It may hold data from e.g. user input and therefore it could be harmful to your application. Three important components described in taint analysis are:

\begin{enumerate}
    \item Source: everything that comes from outside your application scope, like HTTP parameters, user input or file uploads.
    \item Sink: executes an action that may cause damage to your system like SQL queries or functions that interact with your operating system
    \item Sanitizer: protects from untrusted input like esacping html characters or SQL query parameters
\end{enumerate}

If a source can go through a sink without sanitation this is considered a vulnerability. The analysis tool of \textcite[]{Maskur2019} defines these components by using a knowledge base. This base contains a list of variables and functions that are registered as sources, sinks or sanitizers. The knowledge base is a config file within the application \autocite[3]{Maskur2019}. The described system will be an online tool available for developers. The programmer can input source code directly or upload file to be scanned. Then the result will be available to the developers inlcuding the details about the detected vulnerabilities. The security engineers of the tool modify and update the knowledge base of the system.\newline

When the user uploads the source code, the code will be parsed with PHPLY \footnote{ \url{https://github.com/viraptor/phply}} to create an Abstract Syntax Tree (AST). The AST contains nodes including information about the type and the expression of the code. Types can be assignments like variables or functions, switch etc. An example for parsing PHP code to an AST:

\begin{listings}[language=PHP]

    <?php
     $name = $_GET["name"];
     echo "Hello, ". $name;
    ?>     
    
\end{listings}

\begin{verbatim}
('Assignment', 
{'expr': ('ArrayOffset',
    {'expr': u'name', 
    'lineno': 2, 
    'node': ('Variable', {'lineno': 2, 'name': u'$_GET'})}),
'is_ref': False, 
'lineno': 2, 
'node': ('Variable', {'lineno': 2, 'name': u'$name'})}) 
('Echo',
{'lineno': 3, 
'nodes': [('BinaryOp',
    {'left': u'Hello, ', 
    'lineno': 3,
    'op': u'.', 
    'right': ('Variable', {'lineno': 3, 'name': u'$name'})})]})
\end{verbatim}


Taint analysis is applied to the AST and with the data from the knowledge base the tool can detect security vulnerabilities. Unfortunately the tool is limited in terms of object-oriented programming, so there are still false positives detected. \textcite[]{Maskur2019} mentioned the aspect of supporting object-oriented programming and adding more information to the knowledge base as issues for future improvement.


%\begin{comment}
%\subsection{Symbolic Execution Tools}
%\autocite[]{Chaudhuri2010}, \autocite[]{Near2014}, \autocite[]{Near2012}, %\autocite[]{Nijjar2011}, \autocite[]{Near2016}, \autocite[]{Jackson2002}, %\autocite[]{Bocic2016}, \autocite[]{Cadar2011}, \autocite[]{Bocic2014}
%\end{comment}


\section{Dynamic Analysis Tools}
Modern web applications interact with the user and often create content dynamically. Static analysis tools cannot detect vulnerabilities that occur when the application is running. Dynamic tools focus on this aspect and scan web applications at runtime.\newline


\textcite[]{Yip2009} introduced their prototype implementation RESIN for PHP and Python. RESIN uses policy objects and filter object as well as data tracking to prevent access violations and exposure of sensitive data. To use RESIN programmers have to adapt their code accordingly. Programmers themselves define the assertions contained in the policy objects. For example developers can define that a password is only disclosed to the authorized user or the administrator. This policy is then added to the password data. Every I/O channel of the application is observed by filter objects. Data that wants to go through the filters is stopped and the policy object evaluated. If the data is not cleared to pass through an exception is thrown before any data can be exposed \autocite[3-7]{Yip2009}. Another tool that is working with policy objects is GuardRails\autocite[]{Felt2011}. In the Ruby on Rails application the access control policies are directly added to the data model. The output of GuardRails is a new application that enforces those specified policies \autocite[2-4]{Felt2011}.\newline


In contrast to previously mentioned dynamic analysis tools SEPTIC (SElf-ProtecTIng databases from attaCks), developed by \textcite[]{Medeiros2019} prevents malicious attacks from within the database. As the most common attacks include SQL injections or cross-site scripting that both target the database SEPTIC is designed to handle attacks inside the database. SEPTIC is an additional feature of the database. It is implemented in MySQL hence only the backend of the application has to be modified. SEPTIC is not bound to MySQL. It can also be implemented in MariaDB or PostgreSQL, as also recommened in the paper.\autocite[1185]{Medeiros2019}.\newline
The main goal of the implementation was to prevent as the authors call it \textit{semantic mismatch}\autocite[1168]{Medeiros2019}. This refers to ineffective sanitation of input data. How is this possible? Input data is usually sanitized through escaping-functions that escape special characters. Afterwards the sanitized strings is stored in the database. But if attackers use a different encoding of their malicious string, e.g URL encoded ( \%27 instead of `), then the function detects no special characters and cannot sanitize the string. When the database unsanitizes the string before storing it the malicious code found a way into the application despite implemented security features.\newline


How does SEPTIC work? SEPTIC is a module of the database and has three different modes called training mode, detection and prevention mode. In order to setup SEPTIC correctly it needs to observe the normal behaviour of the application. Incoming SQL queries are parsed and validated by the database. SEPTIC analyses the validated queries and creates a query structure. Using this query structure SEPTIC can generate a query model. The query model is the SQL query without the specific input data. It represents the structure and syntactical format. In course of the training mode SEPTIC establishes a query model data store with benign queries of the system\autocite[1169-1172]{Medeiros2019}.\newline


How can SEPTIC detect attacks? SEPTIC compares each validated query it receives with the corresponding query model. If the query doesn't match the query model it is aborted. To ensure there hasn't been any manipulation caused during the parsing of the original query, the query structure that SEPTIC has created is once again compared with the initial request the database has received from the application. Any modification or differences between the initial query, validated query and SEPTIC's query structure can be detected this way. Thus stored malicious values that may be part of a validated query can be detected before execution. Furthermore SEPTIC includes different plugins that scan potentially valid queries also for stored injection attacks. Through structural and syntactical verification SEPTIC can prevent attacks even if unsafe data managed to get stored in the database\autocite[1173-1175]{Medeiros2019}.

\section{Hybrid Analysis Tools}
Static analysis and dynamic analysis have their own individual drawbacks. As static analysis scans the code without executing the program, this means it can analyse the application quicker. But this automatically limits the extent to which the analyser can check the application. Dynamically created content and functions can not be analysed and therefore some vulnerabilities remain undetected. On the other hand dynamic analysis tools often are not provided with the source code and are only monitoring the execution. They rely on attacks vectors to identify vulnerabilities. If those vectors are wrongly constructed this could lead to false positives.
Due to the complex nature of many modern web application hybrid analysis tools that combine static and dynamic elements try to limit the amount of undetected vulnerabilities \autocite[]{Araujo2018, Jahanshahi2018}. In this chapter I would like to provide an insight into a recently developed hybrid analysis tool.\newline


\subsection{SQLBlock}

\textcite[]{Jahanshahi2018} introduce in their recently published paper a tool called SQLBlock that secures a web application against various kinds of SQL Injections. Manipulating SQL queries can lead to serious database damage/manipulation or exposure of classified information to the attacker. An overview of the different categories of SQL attacks by \textcite[3ff.]{Halfond2008} states eight types. Here a short overview of a few categories included.\newline


\textbf{Tautology}: Goal of this attack is to return all rows of the database table. The SQL query is manipulated in a way that the condition stated in the query always returns true. So the manipulation targets the WHERE condition of the query and attackers receive all the data of the table (e.g. adding "or 1=1" to the WHERE-condition)\autocite[3]{Halfond2008}.\newline


\textbf{Union Query}: The attack targets vulnerable parameter with the UNION keyword to increase the amount of data to be returned. This includes data from different tables. By injecting a UNION SELECT inside a parameter the database return the union result of the queries to display more information than intended \autocite[4]{Halfond2008}.\newline


\textbf{Piggy-backed Query}: Unlike the union query attack piggy-backed query attacks try to add multiple queries to the original one. Those attacks are not limited to a specific keyword but can include any type of SQL statement. Therefore this category of manipulation is extremely dangerous and harmful as it can delete and add data arbitrarily \autocite[4]{Halfond2008}.\newline


\textbf{Alternate Encoding}: Standard defense techniques of SQL queries scan for special characters to prevent malicious attacks. To overcome those obstacles attackers make use of other types of encoding to insert their manipulated strings. Different layers of the application handle alternate encoding differently. So a manipulated string could be undetected. As it is almost possible to secure an application against all existing encodings those kinds of attacks have been very successful \autocite[5]{Halfond2008}.\newline

SQLBlock supports object-oriented programming and as mentioned before combines static and dynamic anaylsis methods. The app is focused on PHP applications and is available as a plugin for MySQL and PHP. The prototype implementation of SQLBlock goes through four different phases to evaluate and secure the web application \autocite[1, 5]{Jahanshahi2018}.\newline


In the first phase SQLBlock identifies and analyses the database access layer within the PHP application, this means classes and interfaces that extend the database API like PDO in PHP. Through the static analysis SQLBlock creates a class dependency graph (CDG), that is a directed graph with vertices representing the classes and interfaces, and edges that are drawn if a vertex extends another vertex or implements the interface of the other vertex. After the CDG is fully constructed SQLBlock needs to find all classes and interfaces that are connected to the database API. This extraction leads to creating a list that contains all elements of the application that interact or communicate with the database API and therefore belong to the so called database access layer \autocite[3, 6]{Jahanshahi2018}.\newline


In the next phase SQLBlock is in training mode. During this step SQLBlock is trained to recognize benign SQL queries that happen throughout the application mostly through unit tests. The goal is to create a mapping between the SQL query and the function that constructed the query. To achieve such a mapping first execution information is appended to each query. Then each incoming query is intercepted, SQLBlock records the query as well as the execution information. Through the parse tree of the application SQLBlock can extract information like all included nodes in the query. Furthermore the app also saves the type of SQL operation and the accessed tables of the database.\autocite[6]{Jahanshahi2018}. The result helps to establish a profile that defines all functions that have access to the database. The profile contains query descriptors for each function. Query descriptors contain information about the SQL operation (SELECT, INSERT,...), the current database table, the logical operators that are used in the query and the list of SQL functions that the query applies. These details are based on the previously collected data of the training phase. After the gathering of information and the mapping of functions to benign queries SQLBlock in enforcement mode restricts the access to the database. Each incoming query is checked against the profile and the specific query descriptors. If all components of the descriptor match the profile entry of the query then access to the database is granted \autocite[7]{Jahanshahi2018}.\newline

The capability of SQLBlock to provide protection to the database access of the application is based on the quality of the training phase and the followed mapping. SQLBlock is limited in regards to dynamically generated queries through user input and dynamic PHP features like the eval function. Due to the dynamic nature it is almost impossible to establish a complete profile including all possible SQL queries, which can lead to false positives \autocite[12]{Jahanshahi2018}.

\section{Summary}

% h = try to place the figure Here
% t = try to place the figure at the Top of a page
% p = try to place this figure along with others on a separate Page
% Note that LaTeX has a sophisticated ranking algorithm to place figures.
% It is not always easy to accept LaTeX's placing but it is harder doing it
% manually. Just let it go ;-)

