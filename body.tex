%for reference to this section
\section{Introduction}
\label{section:Introduction}
In recent years web applications have become more and more complex. Due to the growing size of applications it is difficult to satisfy all security requirements. Many applications handle confidential and sensitive data and have become a popular target for malicious attacks. A security breach could have severe consequences depending on the data that has been compromised. The OWASP Foundation\footnote{ \url{https://owasp.org/www-project-top-ten/}} provides a top ten list of web application security risks. SQL Injection, cross-site scripting or sensitive data exposure are only a few of the many issues to consider while developing. But as the implementation of security measures are a very time-consuming job, many developers are lacking the time and/or knowledge to implement those security techniques. Many web applications deployed on the Internet contain severe security vulnerabilities. Almost half of the web applications reviewed by the Web Application Security Consortium contained vulnerabilities that are considered high risk. \autocite[2]{Li2014}
So it comes at no surprise that many security analysis tools have been developed to support developers to scan their applications, reveal bugs or to confirm their security measures they have implemented. There are different types of web security tools. Static analysis tools assess the application without executing the program. A static analysis tool analyses the code to search for e.g. syntatic errors. The analysis considers all code execution paths. Dynamic analysis on the other hand evaluates the app's behaviour during execution. For this kind of analysis specific attack vectors can be used to validate the source code. Some tools combine those techniques. \autocite[]{Lam2008}, \autocite[]{Hosek2011} 
In this paper I would like to give an overview of different types of web security tools. Starting with a more detailed description of the top-ten web security risks and the distinction between static and dynamic analysis in the first chapter. Then I would like to explain in detail how static analysis tools like RubyX, Derailer or SPACE analyse applications \autocite[]{Chaudhuri2010} \autocite[]{Near2014} \autocite[]{Near2016}. A further important topic talking about analysis tools is symbolic execution. Many tools use this analysing technique that translates concrete values in the source code to symbolic values that are executed in an alternate runtime \autocite[]{Bocic2017} \autocite[]{Near2016} \autocite[]{Near2012}. Other tools combine the static and dynamic approach like PQL or SafeWeb or provide black box testing with no access to the source code \autocite[]{Lam2008} \autocite[]{Hosek2011} \autocite[]{Araujo2018}.




\section{Web Application Security Risks}
\autocite[]{ElIdrissi2017}, \autocite[]{Li2014}
\section{Static Analysis vs Dynamic Analysis}
\subsection{Static Analysis Tools}
\autocite[]{Chaudhuri2010}, \autocite[]{Near2014}, \autocite[]{Bocic2014}, \autocite[]{Munetoh2013a}, \autocite[]{Jovanovic2006}, \autocite[]{Munetoh2013}
\subsection{Symbolic Execution Tools}
\autocite[]{Chaudhuri2010}, \autocite[]{Near2014}, \autocite[]{Near2012}, \autocite[]{Nijjar2011}, \autocite[]{Near2016}, \autocite[]{Jackson2002}, \autocite[]{Bocic2016}, \autocite[]{Cadar2011}, \autocite[]{Bocic2014}
\subsection{Dynamic Analysis Tools}
\autocite[]{Yip2009}, \autocite[]{Felt2011}
\subsection{Combining static and analysis Testing}
\autocite[]{Lam2008}, \autocite[]{Hosek2011} \autocite[]{Araujo2018} 


% h = try to place the figure Here
% t = try to place the figure at the Top of a page
% p = try to place this figure along with others on a separate Page
% Note that LaTeX has a sophisticated ranking algorithm to place figures.
% It is not always easy to accept LaTeX's placing but it is harder doing it
% manually. Just let it go ;-)

